\section{Introduction}
In apartments and houses, smart home refers to the employment of technical systems, automated procedures, and connected, remote-controlled gadgets. The functions' major goal is to increase the quality of life and the convenience of the house. Other objectives include improved security and energy efficiency thanks to connected, remote-controllable equipment. What distinguishes a smart house from a typical home? A smart house contains technologies that make our lives easier and more energy efficient. Today's technology includes smart home appliances, mobile devices, and home automation systems, all of which are increasingly networked.

Human will always seek ways to create a more pleasant and also easier life because it is just the nature of human to be at comfortable state and no to forget to help others who seeks for the same thing thus improving our life in general. Smart home which evolved through the progress of home automation was made possible by technical advancements, particularly the Internet and computers. In the 1950s, science fiction literature depicted the first visions of fully automated homes that are watched and controlled by machines. The Disney film "Smart House," released in 1999, was about household computers and the consequences when smart machines take on a life of their own. Disney proved to be unintentionally gave a vision to the audience in one of the scene in the movie where the house’s intelligent control unit develops the feeling of jealousy. Fortunately, it will probably take a few years more before machines can develop and have emotion \cite{b1}.

For decades, a diverse range of home appliances has aided in making routine life more pleasurable, speeding up operations and thus saving time and effort. Yet it has only been in the past 15 years that the issue of the smart home has arised and triggered the interest of public. So, what more advantages does a smart home provide? Without the smart home, humans must initiate each machine's operation by manually initiating procedures and activating each gadget at the right appropriate moment. By allowing components to communicate with one another, the smart home relieves this weight of responsibility of a human.

In this paper, we focus on the system protocol used for the system communication commonly used in smart home and their model of applications in various smart home services. We specifically focus on this category of prediction models and adopt a sequential prediction technique based on text compression algorithms for predicting the occupancy and mobility of the smart home residents. To evaluate the performance of the proposed solutions, a flexible small-scale of smart home is constructed using some of electronic components in order to simulate the system communication implemented. This paper will start by explaining the problem statement of a smart home and the solutions to understand the idea behind this project and also to have some knowledge on the importance of having a smart home. Next, we will be comparing different system communication protocols that commonly used in smart home and explaining on the reasons behind the system communication that we have chosen to be implemented in our project. Finally, we will go through the approach that we took and also the implementation of our project.

