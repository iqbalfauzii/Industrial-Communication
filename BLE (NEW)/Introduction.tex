\section{Introduction}
In apartments and houses, smart home refers to the employment of technical systems, automated procedures, and connected, remote-controlled gadgets. The functions' major goal is to increase the quality of life and ease of use in the house. Other objectives include improved security and energy efficiency thanks to connected, remote-controllable equipment. What distinguishes a smart house from a typical home? A smart house contains technologies that make our lives easier and more energy efficient. Today's technology includes smart home appliances, mobile devices, and home automation systems, all of which are increasingly networked.

It is human nature to find ways that make everyday life easier and more pleasant. The area of home automation in effect the predecessor of the smart home was brought to life through technological progress, in particular through the Internet and computer. Science fiction literature in the 1950s portrayed the first visions of homes that are monitored and controlled fully automatically by machines. The 1999 Disney film “Smart House” was about household computers and the consequences when smart machines take on a life of their own. And Disney proved to be unintentionally visionary in the part of the movie where the house’s intelligent control unit develops the feeling of jealousy. In reality, it will likely be a few years before machines can generate emotions, fortunately.
Scientists have already been working for more than 30 years on connecting home appliances and automating their use. Yet it’s only been in the past 15 years that the issue of the smart home has aroused broad public interest.
At its most basic, a smart home is one that uses so-called “smart” technology to automate and operate important tasks and devices, including lighting, heating and cooling, door locks for home security and not to forget fire alarms to increase home safety. Smart technology is technology that senses what is happening around a particular sensor or device and acts autonomously based on the information it collects. For example, a smart device might sense someone walking into a room and open the shades or turn off the lights or turn up the heat or whatever we have programmed it to do. The goal with these devices is to make your home “smart” enough that we are not bothered by manually performing mundane operations. In this thesis, we focus on prediction models in the smart home and their applications in designing various smart home services. We specifically focus on this category of prediction models and adopt a sequential prediction technique based on text compression algorithms for predicting the occupancy and mobility of the smart home residents. To evaluate the performance of the proposed solutions, a flexible small-scale smart home is constructed using motion sensors and a microcontroller. Several movement scenarios are designed, and the data has been collected by programming the microcontroller and the physical components.
For decades now, a wide range of different home appliances have helped make everyday life more pleasant, speed up processes and hence save time and work. So, what additional benefits does our smart home project deliver? Without the smart home, the impetus for a machine’s every action has to come from humans, who start processes manually and activate each device individually at the right time. The smart home relieves them of this work by enabling components to communicate with each other.
