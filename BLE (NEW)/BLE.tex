

\subsection{Architechture}

The figure "fig.1" best describes the architecture of the BLE protocol. Based on the figure, the BLE protocol uses a stack architecture that is similar to classical Bluetooth. 


\begin{figure}[htbp]
\centerline{\includegraphics{fig1.png}}
\caption{Example of a figure caption.}
\label{fig1}
\end{figure}

The Application layer is the top most layer for the protocol. It is built on top off Generic Access Profile (GAP) and Generic Attribute Profile (GATT). Essentially, it defines how the applications behaviour defined by GAP and GATT.
Generic Access Profile (GAP), Generic Attribute Profile (GATT), Attribute Protocol (ATT), Security Manager (SM) Logical Link Control and Adaptation Protocol (L2CAP) Host Controller Interface (HCI) can be found in host layer

The layers that can be found in the controller are Physical Layer (PHY) and the Link Layer.

The physical layer (PHY) is the hardware level and is used for communication and for some modulation functions for the data to be transferred. BLE operates in the ISM band (2.4 GHz spectrum), which is segmented into 40 RF channels, each separated by 2 MHz (center-to-center), as shown in the figure "fig.2"[1]

\begin{figure}[htbp]
\centerline{\includegraphics{fig1.png}}
\caption{Example of a figure caption.}
\label{fig}
\end{figure}


\subsection{How it works}

The two main concepts to understand in order to start using BLE in projects are Generic Access Profile (GAP) and Generic Attribute Profile (GATT). Both of this component can be found in the host layer. Both of these components, when well defined, is all the users need to start implementing in personal projects.

For GAP, it defines how the device will behave in each manner. The device that utilises BLE protocol can either be a server or a client. Both of this role will be important to establish a Master/slave concept. For a server, the device will advertise itself so that potential compatible devices can detect its presence. For a client, they will scan BLE server to establish a connection. Based on this model, we can conclude that BLE is a point-to-point connection. 
There exists many mode of connection for BLE which include Mesh networks and also Broadcast Mode. This is not covered by this paper because the project will only be using point-to-point connection.

For GATT, it defines how the data is structured. This is best describe in "fig.3" The user must define the GATT well enough at the server side of the communication system. Based on Fig.3, the hierarchy describes how the data that is going to be transferred in the protocol is structured. In the outmost layer we have the profile layer. This layer defines how Services and Characteristics behave. Next we have the service layer which is a collection of characteristics. In Characteristics, we have the data itself. The data have properties for example read write or notify. For our project we will be using write to send commands. Descriptors contain metadata for the data for example the units for temperature.

\begin{figure}[htbp]
\centerline{\includegraphics{fig1.png}}
\caption{Example of a figure caption.}
\label{fig}
\end{figure}

Now for each service and characteristics, there is a UUID which stands for Universally Unique Identifier that serves to distinguish between other services in a BLE network. This ID can be generated online if one cannot be found in the Bluetooth website.

