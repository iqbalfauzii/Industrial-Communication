\section{Comparing Different System Communication}
We know that there are a lot of different system communication that we can implement in smart home and all of them have their own pros and cons in terms of their usability. The best thing to do before implementing our project is to compare the various communication protocols that exist out there. Therefore, we will also be discussing briefly on the advantages and disadvantages of common protocols that are widely used in the system communication of smart home nowadays, which are Wi-Fi, Zigbee, Classic Bluetooth and the brand-new Bluetooth or so-called Bluetooth Low Energy that we choose to be implemented in our project. The comparison has been made by weighing the benefits and drawbacks of these protocols in order to select the best communication protocol. 
\subsection{Classic Bluetooth}
Bluetooth classic is essentially a two-way data transfer protocol. Bluetooth will send wireless data via radio wave, like how Wi-Fi will send data. The difference is that Bluetooth does not require any network equipment such as a modem or router. Bluetooth only requires two enable devices to function [1]. Bluetooth, on the other hand, can only communicate over short distances. For instance, in a range of 100m, which is quite short. In addition to that, Bluetooth classic will use more energy consumption that Bluetooth BLE. Figure 1 shows that main difference between Bluetooth classic and BLE [2].
\subsection{Zigbee}
A very popular communication protocol within the Smart Home community. Zigbee is an Open, flexible (mesh network topology), and low power communication protocol developed on the 2.4 GHz band. It is perfect for battery-based smart home applications but it is not IP-based. As such, Zigbee-based devices require a gateway to connect to the internet for IoT-based applications which increases the cost of deployment. Zigbee offers low bandwidth and sometimes experiences a great deal of interference when deployed alongside WiFi due to competition on the 2.4 GHz band.
\subsection{Wi-fi}
Arguably the most well-known of the bunch, WiFi offers the easiest and probably the most robust communication path for smart home solutions because of its ubiquitous use in other everyday applications. Most homes would already have WiFi routers which makes deployment of WiFi-based smart home devices easier and cheaper. Its high bandwidth makes it suitable for applications that require high data throughput and its IP-based architecture makes deployment for IoT-based applications relatively easier and straightforward compared to other protocols. 

However, all of the goodies come at a cost that includes high power consumption, short-range, and high susceptibility to interference which makes it unsuitable for most battery-powered smart home applications. There have, however, been several improvements over the years, with the most recent version, WiFi 6, offering better power and range performance. 
However, there are some distinct disadvantages of using Wi-Fi as the underlying protocol for your smart home. As the number of Wi-Fi devices grows, the amount of RF interference also grows. How close you live to your neighbors, and your neighbors’ Wi-Fi networks, can impact the performance of your Wi-Fi. Most residential Wi-Fi networks use a single subnet which limits the number of devices to 255. Many Wi-Fi routers can’t even handle this number of simultaneously connected devices. Other than that, Wi-Fi devices connect in a star topology with all devices connecting to a Wi-Fi router, access point, or range extender. This limits how far Wi-Fi devices can be located from the Wi-Fi network in a home.

Other users of Wi-Fi can dramatically affect the performance of a Wi-Fi network. For example, if people in the home start streaming 4K, high definition, video over Wi-Fi, it can dramatically affect the performance of other Wi-Fi devices. Wi-Fi requires more power than other smart home protocols. This decreases the time before battery operated smart home devices need to have their batteries recharged or replaced. Wi-Fi operates at 2.4GHz and 5 GHz. Devices connecting to a Wi-Fi network at 2.4GHz have a practical range of 150 feet. On the other hand, devices connecting to a Wi-Fi network at 5GHz only have a practical range of 50 feet.
\subsection{BLE}
Developed by the Bluetooth special interest group, BLE (Bluetooth Low Energy), also referred to as Bluetooth Smart, is a modification to the Classic Bluetooth protocol with low power consumption as one of its major focuses. It's a product of the desire to reduce the amount of power consumed by devices, both when transmitting data and when idle, to ensure longer battery life. Since it offers simplicity in deployment and low latency with short range, this makes it a perfect choice for smart home.  

